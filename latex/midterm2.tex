\documentclass[12pt]{report}
\usepackage[T2A]{fontenc}
\usepackage[utf8]{inputenc}
\usepackage[ukrainian]{babel}
\usepackage[a4paper, portrait, top=20mm, right=20mm, bottom=20mm, left=20mm]{geometry}
\usepackage[parfill]{parskip}
\usepackage{graphicx}
\usepackage{fancyvrb}
\usepackage{listings}
\usepackage{amsmath}
\usepackage{titlesec}
\usepackage[unicode]{hyperref}
\hypersetup{colorlinks=true, linkcolor=blue, pageanchor=false}
\begin{document}
    \begin{titlepage}
        \centering
        Київський академічний університет
        \vfill
        \vspace{0.5cm}
        \huge \textbf{Задача планування виробництва} \\
        \vspace{0.5cm}
        \normalsize Данило Петраківський
        \vfill
        \normalsize Київ --- 2023
    \end{titlepage}

    \section*{Постановка задачі}\label{sec:problem}

    Фірма на запланований період має виготовити $a$ одиниць продукції, використовуючи $n$ технологічних способів.
    При виробництві $x_j$ одиниць продукції $j$-м технологічним способом витрати рівні
    $c_j x_j + d_j x_j ^ 2 (c_j > 0, d_j > 0)$.
    Обсяги використовуваних ресурсів $i$-го виду $(i = \overline{1, m})$ та норми їх використання $j$-м технологічним
    способом відповідно дорівнюють $b_i$ і $a_{ij}$.
    Визначити скільки виробів треба виготовити кожним способом, щоб загальні витрати на виробництво були мінімальними.
    Математична модель цього завдання зводиться до задачі нелінійного (квадратичного) програмування:

    \[\min z = \sum_{j = 1}^{n} (c_j x_j + d_j x_j ^ 2)\]

    При обмеженнях:

    \begin{gather*}
        \sum_{j = 1}^{n} x_j = a \\
        \sum_{j = 1}^{n} a_{ij} x_j \le b_i, (i = \overline{1, m}) \\
        x_j \ge 0, (j = \overline{1, n})
    \end{gather*}

    \section*{Вхідні дані}\label{sec:input}

    У якості зразка візьмемо невелику кількість типів продукції і ресурсів, $a$:

    \begin{center}
        \begin{tabular}{c c c}
            2.567 & 7.21  & 0.12 \\
            1.98  & 7.3   & 6.21 \\
            0.31  & 2.721 & 8.12 \\
            1.372 & 4.203 & 2.91
        \end{tabular}
    \end{center}

    Відповідні їм значення норм, $b$:

    \begin{center}
        \begin{tabular}{c c c c}
            95 & 150.67 & 23.76 & 61.21
        \end{tabular}
    \end{center}

    Витрат першого порядку $c$:

    \begin{center}
        \begin{tabular}{c c c}
            0.6 & 2.6 & 9.15
        \end{tabular}
    \end{center}

    Витрат другого порядку $d$:

    \begin{center}
        \begin{tabular}{c c c}
            0.19 & 0.08 & 1.5
        \end{tabular}
    \end{center}

    У сумарно слід виготовити $a = 11$ пристроїв.
    В оптимальному випадку слід виготовити 7 речей першого типу й 4 другого.

    \section*{Програмні реалізації}\label{sec:implementations}

    Весь код написаний і зневаджений на мові python з використанням готових рішень від scipy та самописних функцій.

    \subsection*{SciPy, trust-region}\label{subsec:scipy-trust-region}

    Для початку, реалізуємо вирази для обчислення функції, якобіана та гессіана:

    \begin{lstlisting}[language=Python,label={lst:scipy-trust-region1}]
from numpy import ndarray, diag

def planning_objective(
    x: ndarray,
    c: ndarray,
    d: ndarray
) -> float:
    return c @ x + d @ x ** 2


def planning_jacobian(
    x: ndarray,
    c: ndarray,
    d: ndarray
) -> ndarray:
    return c + 2 * d * x


def planning_hessian(
    _x: ndarray,
    _c: ndarray,
    d: ndarray
) -> ndarray:
    return diag(2 * d)
    \end{lstlisting}

    А тепер сама процедура обчислення на основі
    \href{https://docs.scipy.org/doc/scipy/reference/optimize.minimize-trustconstr.html}{довірчої області}:

    \begin{lstlisting}[language=Python,label={lst:scipy-trust-region-2}]
from numpy import inf, zeros, full, vstack, ones, insert
from scipy.optimize import (
    Bounds,
    LinearConstraint,
    OptimizeResult,
    minimize
)

def minimize_planning_constrained(
    c: ndarray,
    d: ndarray,
    s: float,
    a: ndarray,
    b: ndarray
) -> OptimizeResult:
    m, n = len(b), len(c)
    x0 = full(n, s // n)
    x0[: s % n] += 1
    bounds = Bounds(zeros(n), full(n, inf))
    constraint = LinearConstraint(
        vstack([ones((1, n)), a]),
        insert(zeros(m), 0, s, 0),
        insert(b, 0, s, 0)
    )
    return minimize(
        planning_objective,
        x0,
        (c, d),
        'trust-constr',
        planning_jacobian,
        planning_hessian,
        bounds=bounds,
        constraints=constraint
    )
    \end{lstlisting}

    На що отримуємо дуже хороші результати роботи й майже вірну відповідь:

    \begin{verbatim}
 barrier_parameter: 0.0008000000000000003
 barrier_tolerance: 0.0008000000000000003
          cg_niter: 9
      cg_stop_cond: 1
            constr: [array([11., 46.98129589, 43.25837261, 13.14470093,
                    26.52146586]), array([6.96268412e+00, 4.03718023e+00,
                    1.35645694e-04])]
       constr_nfev: [0, 0]
       constr_nhev: [0, 0]
       constr_njev: [0, 0]
    constr_penalty: 1.0
  constr_violation: 0.0
    execution_time: 0.01324915885925293
               fun: 25.19043053586868
              grad: array([3.24581997, 3.24594884, 9.15040694])
               jac: [array([[1.   , 1.   , 1.   ],
       [2.567, 7.21 , 0.12 ],
       [1.98 , 7.3  , 6.21 ],
       [0.31 , 2.721, 8.12 ],
       [1.372, 4.203, 2.91 ]]), array([[1., 0., 0.],
       [0., 1., 0.],
       [0., 0., 1.]])]
   lagrangian_grad: array([ 5.31606926e-10, -6.61158041e-10,  1.29550592e-10])
           message: '`gtol` termination condition is satisfied.'
            method: 'tr_interior_point'
              nfev: 8
              nhev: 8
               nit: 11
             niter: 11
              njev: 8
        optimality: 6.611580405336756e-10
            status: 1
           success: True
         tr_radius: 5526.706993994392
                 v: [array([-3.24567701e+00, -3.67875360e-07,
                    -1.10455447e-05, 1.45019076e-05, -7.10186596e-06]),
                    array([-1.14898229e-04, -1.98158068e-04, -5.90475838e+00
                    ])]
                 x: array([6.96268412e+00, 4.03718023e+00, 1.35645694e-04])
    \end{verbatim}

    \subsection*{SciPy, штрафна функція}\label{subsec:scipy-penalty-objective}

    Тут ми обираємо алгоритм BFGS, але перед тим реалізуємо відповідні функцію, якобіан і гессіан:

    \begin{lstlisting}[language=Python,label={lst:scipy-penalty-objective-1}]
from numpy import maximum, where

def planning_with_penalty_objective(
    x: ndarray,
    c: ndarray,
    d: ndarray,
    s: float,
    a: ndarray,
    b: ndarray,
    mu: float = 1e6
) -> float:
    alpha = (
        sum(maximum(a @ x - b, 0) ** 2)
        + sum(maximum(-x, 0) ** 2)
        + (sum(x) - s) ** 2
    )
    return planning_objective(x, c, d) + mu * alpha


def planning_with_penalty_jacobian(
    x: ndarray,
    c: ndarray,
    d: ndarray,
    s: float,
    a: ndarray,
    b: ndarray,
    mu: float = 1e6
) -> ndarray:
    g = repeat(reshape(a @ x - b, (-1, 1)), len(c), 1)
    alpha = 2 * sum(maximum(g, 0) * where(g >= 0, a, 0), 0)
    alpha += (
        2 * maximum(-x, 0) * where(x <= 0, -1, 0)
        + 2 * (sum(x) - s)
    )
    return planning_jacobian(x, c, d) + mu * alpha
    \end{lstlisting}

    І сама процедура мінімізації:

    \begin{lstlisting}[language=Python,label={lst:scipy-penalty-objective-2}]
def minimize_planning_with_penalty(
    c: ndarray,
    d: ndarray,
    s: float,
    a: ndarray,
    b: ndarray
) -> OptimizeResult:
    n = len(c)
    x0 = full(n, s // n)
    x0[: s % n] += 1
    return minimize(
        planning_with_penalty_objective,
        x0,
        (c, d, s, a, b),
        'BFGS',
        planning_with_penalty_jacobian
    )
    \end{lstlisting}

    Отримані результати процесу:

    \begin{verbatim}
      fun: 25.189618281111358
 hess_inv: array([[ 1.85185194e+00, -1.85185164e+00, -1.48147908e-07],
       [-1.85185164e+00,  1.85185235e+00, -3.51851285e-07],
       [-1.48147908e-07, -3.51851285e-07,  4.99999222e-07]])
      jac: array([-5.23889865e-09, -5.23889687e-09, -5.23861488e-09])
  message: 'Optimization terminated successfully.'
     nfev: 99
      nit: 18
     njev: 99
   status: 0
  success: True
        x: array([ 6.96296336e+00,  4.03703797e+00, -2.95203253e-06])
    \end{verbatim}

    І трохи даних про штрафну область, а саме — значення, яких набуває модифікована функція поза обмеженнями.
    Наприклад, перевищення сумарної кількості виробів (точка $(5, 5, 10)$) дає штраф $5351034289.249998$, нестача
    (точка $(0, 0, 0)$) дає $121000000$, а надмірна кількість предметів третього типу (точка $(1, 1, 9)$) повертає
    штраф $2740627408.3199987$.

    \subsection*{ralgb5a, штрафна функція}\label{subsec:ralgb5a-penalty-objective}

    Тут ми скористаємося раніше
    \href{https://github.com/xXxRisingTidexXx/optima-labs/blob/62f605afb3c3caf3be5f4a18b1b14d9a99404bd4/optimalabs/nlp.py}{реалізованим r-алгоритмом}
    та відповідними процедурами задачі, якобіана й гессіана зі штрафними функціями для безумовної оптимізації.

    \begin{lstlisting}[language=Python,label={lst:ralgb5a-penalty-objective}]
from optimalabs.nlp import Func, ExitCode, ralgb5a

SUCCESS_CODES = {ExitCode.eps_f, ExitCode.eps_g, ExitCode.eps_x}


def minimize_planning_ralgb5a(
    c: ndarray,
    d: ndarray,
    s: float,
    a: ndarray,
    b: ndarray
) -> OptimizeResult:
    n = len(c)
    x0 = full((n, 1), s // n)
    x0[: s % n] += 1
    result = ralgb5a(
        make_planning_with_penalty_joint(c, d, s, a, b),
        x0
    )
    return OptimizeResult(
        x=result.x_r.reshape(-1),
        success=(result.exit_code in SUCCESS_CODES),
        status=result.exit_code.value,
        message=result.exit_code.name,
        fun=planning_with_penalty_objective(
            result.x_r,
            c,
            d,
            s,
            a,
            b
        ),
        nit=result.n_iter,
        nfev=result.n_calls,
        njev=result.n_calls
    )


def make_planning_with_penalty_joint(
    c: ndarray,
    d: ndarray,
    s: float,
    a: ndarray,
    b: ndarray,
    mu: float = 1e6
) -> Func:
    def func(x: ndarray) -> tuple[float, ndarray]:
        x0 = x.reshape(-1)
        f = planning_with_penalty_objective(x0, c, d, s, a, b, mu)
        g = planning_with_penalty_jacobian(x0, c, d, s, a, b, mu)
        return f, g.reshape((-1, 1))

    return func
    \end{lstlisting}

    І результати роботи, близькі до правильних:

    \begin{verbatim}
     fun: array([9.26969805e+08])
 message: 'eps_x'
    nfev: 130
     nit: 61
    njev: 130
  status: 3
 success: True
       x: array([ 6.96296336e+00,  4.03703797e+00, -2.95203252e-06])
    \end{verbatim}

    \section*{Висновки}

    \begin{enumerate}
        \item Метод штрафних функцій дозволяє обчислювати оптимальне значення з точністю, що близька до перевірених
        методів, як от довірчих областей.
        \item r-алгоритм потребує більш, ніж у 10 разів більше ітерацій за стандартизовані реалізації.
        \item Точна реалізація гессіана зменшує кількість ітерацій методу довірчих областей на ~45\%.
        \item Наведені алгоритми повертають дійсні значення, але результати мають належати множині цілих чисел; для
        доведення відповіді слід застосувати, наприклад, повний перебір між округленими значеннями, щоб відшукати
        мінімум цільової функції.
    \end{enumerate}
\end{document}
